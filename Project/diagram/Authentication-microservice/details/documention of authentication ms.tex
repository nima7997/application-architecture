\documentclass[a4paper]{report}

\usepackage{fancyhdr}
\usepackage{graphicx}
\usepackage{xepersian}

\fancyhead[R]{خواجه نصیر الدین طوسی}
\fancyhead[L]{سید حسین زراعتکار}
\settextfont{B Nazanin}
\setlatintextfont{Times New Roman}

\title{ معماری نرم‌افزار - سرویس \lr{Authentication} }
\author{سید حسین زراعتکار و نیما گمرکیان\\
	دانشگاه خواجه نصیر الدین طوسی}
\date{\today}

\begin{document}
	\pagestyle{plain}
\begin{figure}
	\centering
	\includegraphics[width=0.7\linewidth]{image/KNTU.png}
	\label{fig:kntu}
\end{figure}
\maketitle
\section{میکروسرویس \lr{Authentication} - \lr{Module View}}
در این بخش به بررسی معماری میکروسرویس \lr{َAuthentication} خواهیم پرداخت. این میکروسرویس وظیفه احراز هویت کاربران و ادمین‌ها را به عهده دارد. پس از اهراز هویت به هر کاربر یک توکن داده خواهد شد که دارای زمان انقضا هست. هر سرویس دیگر متناسب با توکن‌های کاربران به آنها سرویس می‌دهد. در ادامه ماژول‌ها و وابستگی‌شان را در قالب دیاگرام در شکل \ref{fig:authentication-diragram} ارائه کردیم.
\begin{figure}[h]
	\centering
	\includegraphics[width=1.0\linewidth]{image/Authentication-Diragram}
	\caption{دیاگرام وابستگی ماژول‌ها}
	\label{fig:authentication-diragram}
\end{figure}

	این ماژول از طریق دو \lr{framework} ‌‌با نام‌های \lr{grpc} و \lr{RabitMQ} با دیگر سرویس‌ها و \lr{bff} تعامل می‌کند. این قسمت در ماژولی مجزا به نام \lr{sidebar-proxy} استفاده می‌شود و دلیل آن استفاده از \lr{mesh service pattern}   می‌باشد. این الگو سبب بهبود کارایی می‌شود و برای انجم کارهای شبکه و پیاده سازی موارد امنیتی ورودی و خروجی و رمزگذاری استفاده می‌شود. اگرچه با افزودن این لایه باعث مصرف بیشتر توان می‌شود و سربار دارد. برای تعاملات بین سرویس به دلیل سرعت بالاتر از \lr{grpc} و برای تعاملات با \lr{bff} از \lr{RabitMQ} استفاده کردیم.\\
	برای پیاده سازی متد‌ها و مدیریت پیام‌های این دو سرویس یک ماژول پیاده سازی کردیم که بتواند کارهایی که فریموورک قرار است برای ما انجام دهد را مدیریت کند که این ماژول باید \lr{interface} مربوط به \lr{message interface} را پیاده سازی کند تا \lr{business logic} ما بتواند بدون وابستگی به ماژول \lr{low level} با بقیه ارتیاط برقرار کند. در قسمت \lr{business logic} ما یک ماژول به منظور هندل کردن رخدادهای دریافتی یا ارسال پیام‌ها پیاده سازی شده است به نام \lr{Message Handler} این ماژول نیز به علت سطح پایین بودن باید برای مورد استفاده قرار گرفتن توسط \lr{Authentication} که بالاترین سطح را دارد با استفاده از معکوس سازی وابستگی و استفاده از \lr{interface} پیاده سازی شود. از طرفی دیگر ما برای افزودن کاربران(ادمین و کاربر‌معمولی) ماژولی به نام \lr{Management User Service} داریم که عملیات های \lr{verification, registration, update} را انجام می‌دهد. این ماژول نیز به دلیل نیاز به استفاده از دیتابیس با \lr{interface} و معکوس سازی وابستگی با دیتابیس مورد نظر کار می‌کند. در آخر ماژول \lr{Tokenizer service} با استفاده از کتابخانه‌ها و ‌\lr{framework} ها توکن‌ها را ایجاد کرده و آپدیت و انقضای توکن‌ها از این طریق مدیریت می‌شوند. برای انقضای توکن در واسط مربوط به \lr{Tokenizer} ما یک متد برای ایونت انقضا داشته و در طرف مازول \lr{mangement} یک هندلر تابع و در \lr{authentication} این‌ها را مدیریت می‌کنیم.
\begin{enumerate}
	\item یکی از نکات مهم این است که در سرویس‌های دیگر برای تعاملات باید از معماری تراکنش \lr{saga} استفاده کنیم تا قبل از اعمال تراکنش احراز هویت بر اساس توکن کاربر در این سرویس بررسی شود.
	\item نکته دیگر این است که برای تست پذیری از قابلیت ذخیره سازی دادگان در فایل نیز انجام می‌شود و البته نیاز به رعایت نکاتی برای امنیت و کارایی برنامه در هنگام پیاده سازی دارد. (َ\lr{Testability: Abstract Data source})
	\item همچنین به منظور تست پذیری بهتر زمان پیاده سازی واسط‌ها باید متدهایی به منظرو تست آن بخش پیاده سازی شود. (\lr{Testability: Specialize Interface})
	\item به دلیل استفاده از معماری میکروسرویس ما قابلیت تست پذیری بیشتری داشته‌ایم زیرا سرویس‌ها مجزا شده اند و \lr{separation of concern} را رعایت کرده و وابستگی درو ماژولی افزایش و وابستگی بین ماژولی کاهش یافته است. \lr{(Testability: Limit structural complexity)}
	\item بر اساس فریمورک‌هایی مانند \lr{jwt} نیاز به اهراز هویت توکن برای هر تراکنش نیست یعنی با این سرویس نیاز نیست که اهراز هویت توکن را هربار چک کنیم ولی نیاز به \lr{verify} داریم. بر اساس \lr{public key} که به سرویس‌ها ارسال شده اهراز هویت هر تراکنش انجام می‌شود.
\end{enumerate}
	
	
	
	
		
\section{میکروسرویس \lr{Authentication} - \lr{Component \& Connector View (C\&C)}}
در این قسمت به بررسی نمای دیگر از معماری سرویس \lr{Authentication} به نام \lr{Connection And Component View} می‌پردازیم. نمای کلی ارتباطات این سرویس با سرویس‌های دیگر و \lr{BFF} در شکل \ref{fig:support-diagram} ارائه شده است.
\begin{figure}[h]
	\centering
	\includegraphics[width=0.7\linewidth]{"image/Authentication-C&C View-Diragram.drawio"}
	\caption{}
	\label{fig:authentication-cc-view-diragram}
\end{figure}

بر اساس نیاز موجود هر کاربر با هر سطح دسترسی یا درخواست اهراز هویت اولیه می‌دهد که در آن صورت مسیر از کاربر به \lr{BFF}، \lr{API Gateway} و در آخر به سرویس احراز هویت میرسد و اما در صورت درخواست سرویس دیگر مرحله آخر به سرویس مورد نظر رسیده و سرویس مورد نظر با کلید عمومی دریافت شده از سرویس \lr{Authentication} میتواند تراکنش را اهراز هویت کند. ضمننا، پس از تاریخ انقضای هر توکن، توکن نامعتبر در هر سیستم می‌باشد.\\
برای دسترس پذیری سرویس \lr{Authentication} از متد \lr{Heartbeat (warm standby)} استفاده کردیم. به دلیل تغییرات کم در کاربران چندوقت یکبار سرویس کمکی همگام سازی می‌شود. با خرابی یکی از سرویس‌ها دیگری سریعا جایگزین می‌شود.
\end{document}}