\documentclass[a4paper]{report}

\usepackage{fancyhdr}
\usepackage{graphicx}
\usepackage{xepersian}

\fancyhead[R]{خواجه نصیر الدین طوسی}
\fancyhead[L]{سید حسین زراعتکار}
\settextfont{B Nazanin}
\setlatintextfont{Times New Roman}

\title{ معماری نرم‌افزار - سرویس \lr{Admin/Support} }
\author{سید حسین زراعتکار و نیما گمرکیان\\
	دانشگاه خواجه نصیر الدین طوسی}
\date{\today}

\begin{document}
	\pagestyle{plain}
\begin{figure}
	\centering
	\includegraphics[width=0.7\linewidth]{image/KNTU.png}
	\label{fig:kntu}
\end{figure}
\maketitle
\section{میکروسرویس \lr{ُSupport\ Admin} - \lr{Module View}}
این سرویس توانایین \lr{CRUD} روی هر سرویسی را دارد. پس از طریق دو فریموورک \lr{grpc \& RabbitMQ} می‌تواند با سایر سرویس ها و \lr{BFF} ارتباط برقرار کند. به ازای هر سرویس موجود یک \lr{manager module} داریم که در صورت دریافت فرمان از \lr{BFF} با سطح دسترسی ادمین می‌تواند سرویس مورد نظر را تغییرات دهد. همچنین به منظور مدیریت رخداد‌ها و امکان \lr{Rollback} رخدادها را در دیتابیس ذخیره می‌نماییم. ماژول \lr{inventory} بر روی سرویس‌ \lr{Authentication} به دلیل نام کاربری و رمزعبور فروشگاه تاثیر گذار است پس برخی تراکنش‌ها نیاز به اعمال \lr{saga} دارد. همچنین ماژول \lr{Accounting} بر \lr{inventory} نیز ممکن است تغییراتی ایجاد کند زیرا با فاکتورهای جدید موجودی نیز تغییر می‌کند. به دلیل اینکه تعداد ادمینها محدود است امکان لود بالا در این سرویس نبوده و دسترس پذیری را خواهیم داشت و همچنین در صورت ایجاد مشکل شرایط باقی سرویس‌ها به مشکل بر نخواهد خورد. کارایی ایت سروی نیز به دلیل مشخص بودن حدود تعداد ادمین‌ها می‌تواند در ابتدا در انتخاب قدرت سیستم تصمیم گیری شود. تست پذیری به دلیل استفاده از میکروسرویس‌ها بالا رفته زیرا نگرانی ها را جداسازی کردیم و وابستگی درون ماژولی و کاهش وابستگی به ماژول خارجی را کاهش دادیم.
با افزایش کاربران این سرویس مستقیما تحت تاثیر قرار نخواهد گرفت پس مقیاس پذیری در این بخش نیاز نخواهیم داشت. دیاگرام وابستگی مازولی این سرویس در شکل \ref{fig:untitled-diagram} ارائه شده است.
\begin{figure}[h]
	\centering
	\includegraphics[width=1\linewidth]{"image/Untitled Diagram.drawio"}
	\caption{\lr{Module View} برای سرویس \lr{Support\ Admin}}
	\label{fig:untitled-diagram}
\end{figure}

\section{میکروسرویس \lr{Support\ Admin} - \lr{C\&C}}
در این قسمت به بررسی جریان کاری برای استفاده از این سرویس خواهیم پرداخت. ادمین با استفاده از \lr{BFF} به سرویس \lr{Authentication} متصل شده و وارد حساب خود می‌شود. توکن ادمین در صورت اهرازهویت به سرویس ادمین داده خواهد شد. حال دستورات ارسال شده به سرویس ادمین پس از صحت سنجی توکن به باقی سرویس ها ارسال شده و نتایج آن را گرفته و به ادمین ارسال می‌کند. سرویس ادمین توانایی کاربر معمولی(فروشگاه)  را خواهد داشت همچنین قابلیت گزارش‌گیری تغییرات در فروشگاه، فاکتورها و کالا‌ها و همچنین مدیریت کاربران و ادمین ها را خواهد داشت. نمودار این بخش در شکل \ref{fig:cc-supportadmin-diagram} ارائه شده است.
\begin{figure}
	\centering
	\includegraphics[width=1\linewidth]{image/C&C-SupportAdmin-diagram.drawio}
	\caption{\lr{C\&C View} برای سرویس \lr{Support\ Admin}}
	\label{fig:cc-supportadmin-diagram}
\end{figure}

\end{document}}