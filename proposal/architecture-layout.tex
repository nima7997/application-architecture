\documentclass[12pt,a4paper]{article}
\usepackage{geometry}
\geometry{a4paper, margin=2.5cm}
\usepackage{graphicx}
\usepackage{fancyhdr}
\usepackage{xepersian}
\settextfont{B Nazanin}
\usepackage{titlesec}
\usepackage{enumitem}
\usepackage{lipsum}
\usepackage{float}


\pagestyle{fancy}
\fancyhf{}
\fancyhead[L]{\textbf{نام اعضای تیم:} سید حسین زراعتکار و نیما گمرکیان}
\fancyhead[R]{\textbf{موضوع پروژه:} معماری نرم‌افزار فروشگاه‌های زنجیره‌ای}
\fancyfoot[C]{\thepage}


\titleformat{\section}{\large\bfseries}{\thesection}{1em}{}
\titleformat{\subsection}{\normalsize\bfseries}{\thesubsection}{1em}{}


\usepackage{xcolor}
\usepackage[
colorlinks=true,
linkcolor=black,    
urlcolor=blue,       
citecolor=black
]{hyperref}


\makeatletter
\renewcommand{\@dotsep}{4.5}
\makeatother

\begin{document}
	\begin{figure}
		\centering
		\includegraphics[width=0.7\linewidth]{Image/Icon/KNTU}
		\label{fig:kntu}
	\end{figure}
	\begin{titlepage}
		\centering
		\vspace*{3cm}
		{\Huge \textbf{پلتفرم  مدیریت  فروشگاه‌های زنجیره‌ای}\par}
		\vspace{2cm}
		{\Large \textbf{درس:} معماری نرم‌افزار \par}
		\vspace{1cm}
		{\Large \textbf{استاد:} دکتر اثنی‌عشری \par}
		\vspace{1cm}
		{\Large \textbf{اعضای تیم:} سید حسین زراعتکار و نیما گمرکیان\par}
		\vfill
		{\large تاریخ: \today \par}
	\end{titlepage}
	

	\newpage
	

	\section{مقدمه}
	این سامانه به منظور استفاده در فروشگاه‌های زنجیره‌ای برای مدیریت کارمندان و مشتریان، مدیریت محصولات و موجودی، مدیریت فروش(تراکنش‌ها) و پرداخت طراحی خواهد شد. این سامانه نیاز به یکپارچه‌سازی به منظور مدیریت شعب، مدیریت فروش، مدیریت موجودی، مدیریت پرسنل و مشتریان دارد. این نرم‌افزار دارای دو سطح دسترسی کاربر و مدیر می‌باشد. در هر یک از شعب باید قابلیت ویرایش موجودی کالا، ویرایش یا افزودن مشتری جدید، ثبت و ویرایش فاکتور فروش وجود داشته باشد. تعریف پرسنل، کالاها، سطح دسترسی (کاربر و مدیر )توسط نرم افزار مدیریتی انجام می‌شود.
	\section{موارد استفاده}
	سرویس فروشگاه برای هر شعبه
	\begin{enumerate}
	\item اپراتور در نرم‌افزار فروشگاه با سطح دسترسی کاربر معمولی وارد می‌شود.
	\item احرازهویت ایشان توسط سرویس احرازهویت تصدیق می‌شود.
	\item اپراتور انتخاب‌های زیر را خواهد داشت:
	\begin{enumerate}
%	\item افزودن و ویرایش مشتری جدید	
	\item افزودن، ویرایش، و حذف اقلام فاکتور فروش()(\lr{inventory})
	\item ویرایش موجودی کالاها(\lr{inventory})
	\end{enumerate}
	\end{enumerate}
	سرویس مدیریتی
	\begin{enumerate}
		\item کاربر با سطح دسترسی مدیر وارد می‌شود.
		\item احرازهویت وی توسط سرویس احرازهویت تصدیق می‌شود.
		\item مدیر قابلیت‌های زیر را خواهد داشت:
		\begin{enumerate}
			\item افزودن،ویرایش و حذف اپراتور(\lr{authentication})
			\item افزودن، ویرایش و حذف کالا،موجودی،قیمت و .. (\lr{inventory})
			\item گزارشگیری‌های متنوع (\lr{report,admin})
			\item تمام توانایی کاربر معمولی (\lr{inventory, accounting})
		\end{enumerate}
	\end{enumerate}
	%\textbf{حذف و ویرایش رکورد برای هر داده براساس شرایطی خاص انجام می‌شود.} \\
	دیگر سرویس‌ها که شامل سرویس مدیریت کاربران و مشتریان، سرویس مدیریت فروش و موجودی و کالا و مدیریت گزارش‌ها می‌شود توسط این دو سرویس مورد استفاده قرار خواهند گرفت.
	
	برای این نرم‌افزار از معماری میکروسرویس استفاده خواهد شد. تمرکز بر معماری مبکروسرویس‌های 
\lr{Authentication},
\lr{Support},
\lr{Inventory},
\lr{Accounting},
\lr{Report}
می‌باشد.
	\section{پیشرانه‌های معماری (\lr{Architecture Drivers})}
	پیشرانه‌های مدنظر:
	\begin{enumerate}
		\item \textbf{\lr{availability}}
		%سرویس‌دهی با بیشترین دسترس‌پذیری (به ویژه در ساعات اوج تراکنش‌ها)
		\item \textbf{\lr{performance}}
	%	پاسخ درخواست‌ها در بازه محدود و مشخص داده شود
		\item \textbf{\lr{testability}}
	%	قابلیت نمایش خطای سیستم برای تسهیل در تصحیح خطای
		
	\end{enumerate}
	برخی از پیشرانه‌های دیگر که می‌توان در نظر گرفت:
	\begin{enumerate}
		\item \textbf{\lr{security}}
		\item \textbf{\lr{energy efficiency}}		
		\item \textbf{\lr{usability}}
		
		
	\end{enumerate}
	\newpage
	\section{سناریوها (\lr{Scenarios})}
	\textbf{\lr{availability scenario}}
	\begin{enumerate}
		\item سامانه باید در ساعات کاری به برای انجام عملیات‌هایی همچون ثبت تراکنش‌ها در دسترس باشد.
		\begin{table}[h]
			\centering
			\caption{دسترس‌پذیری} % heart beat, redundant spare,retry,graceful degradation
			\begin{tabular}{|c|c|}
				\hline
				\textbf{بخش‌های سناریو} & \textbf{توضیحات} \\
				\hline
				\lr{stimulus sources} & \lr{salesclerk or support user, services}\\
				\lr{stimulus} & \lr{crash, fault, disconnect network} \\
				\lr{environment} &  \lr{peak hours} \\
				\lr{artifact} & \lr{hardware, network, micro-services, api gateway}\\
				\lr{response} & \lr{log the fault, notify, recover from fault}  \\
				\lr{response measures} & \lr{time to detect fault, availability percentage, time to recover} \\
				\hline
			\end{tabular}
		\end{table}
	\end{enumerate}
\textbf{\lr{performance scenario}}
\begin{enumerate}
	\item در زمان اوج فروش در شعب باید حداکثر تاخیر قابل قبولی داشته باشیم.
	\item در زمان مطلوبی بتوان نتایج گزارشات را دریافت کرد.
\end{enumerate}
\begin{table}[h]
	\centering
	\caption{کارآیی}
	\begin{tabular}{|c|c|}
		\hline	
		\textbf{بخش‌های سناریو} & \textbf{توضیحات} \\
		\hline	
		\lr{stimulus sources} & \lr{salesclerk or support user. services}\\
		\lr{stimulus} & \lr{transaction, report} \\
		\lr{environment} &  \lr{peak time,normal time} \\
		\lr{artifact} & \lr{inventory and accounting micro-services,report micro-service}\\
		\lr{response} & \lr{return response or Error , ignore request }  \\
		\lr{response measures} & \lr{response time, latency} \\
		\hline
	\end{tabular}
\end{table}
\textbf{\lr{testability scenario}}
\begin{enumerate}
	\item درصورت بروز خطا در سیستم قابلیت کشف علت و تصحیح آن قبل از استقرار مشخص باشد.	
\end{enumerate}
\begin{table}[h!]
	\centering
	\caption{تست‌پذیری}
	\begin{tabular}{|c|c|}
		\hline	
		\textbf{بخش‌های سناریو} & \textbf{توضیحات} \\
		\hline	
		\lr{stimulus sources} & \lr{developer and tester ,unit test}\\
		\lr{stimulus} & \lr{validate system functions} \\
		\lr{environment} &  \lr{design \& development \& compile \& integration time} \\
		\lr{artifact} & \lr{each portion of system which we want to test}\\
		\lr{response} & \lr{execute test, capture state report that each module tell us}  \\
		\lr{response measures} & \lr{effort to find a fault or class of faults} \\
		\hline
	\end{tabular}
\end{table}


\end{document}
