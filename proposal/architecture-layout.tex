\documentclass[12pt,a4paper]{article}
\usepackage{geometry}
\geometry{a4paper, margin=2.5cm}
\usepackage{graphicx}
\usepackage{fancyhdr}
\usepackage{xepersian}
\settextfont{B Nazanin}
\usepackage{titlesec}
\usepackage{enumitem}
\usepackage{lipsum}
\usepackage{float}


\pagestyle{fancy}
\fancyhf{}
\fancyhead[L]{\textbf{نام اعضای تیم:} سید حسین زراعتکار و نیما گمرکیان}
\fancyhead[R]{\textbf{موضوع پروژه:} معماری نرم‌افزار فروشگاه‌های زنجیره‌ای}
\fancyfoot[C]{\thepage}


\titleformat{\section}{\large\bfseries}{\thesection}{1em}{}
\titleformat{\subsection}{\normalsize\bfseries}{\thesubsection}{1em}{}


\usepackage{xcolor}
\usepackage[
colorlinks=true,
linkcolor=black,    
urlcolor=blue,       
citecolor=black
]{hyperref}


\makeatletter
\renewcommand{\@dotsep}{4.5}
\makeatother

\begin{document}
	\begin{figure}
		\centering
		\includegraphics[width=0.7\linewidth]{Image/Icon/KNTU}
		\label{fig:kntu}
	\end{figure}
	\begin{titlepage}
		\centering
		\vspace*{3cm}
		{\Huge \textbf{پلتفرم  مدیریت  فروشگاه‌های زنجیره‌ای}\par}
		\vspace{2cm}
		{\Large \textbf{درس:} معماری نرم‌افزار \par}
		\vspace{1cm}
		{\Large \textbf{استاد:} دکتر اثنی‌عشری \par}
		\vspace{1cm}
		{\Large \textbf{اعضای تیم:} سید حسین زراعتکار و نیما گمرکیان\par}
		\vfill
		{\large تاریخ: \today \par}
	\end{titlepage}
	

	\newpage
	

	\section{مقدمه}
	این سامانه به منظور استفاده در فروشگاه‌های زنجیره‌ای برای مدیریت کارمندان و مشتریان، مدیریت محصولات و موجودی، مدیریت فروش(تراکنش‌ها) و پرداخت طراحی خواهد شد. این سامانه نیاز به یکپارچه‌سازی به منظور مدیریت شعب، مدیریت فروش، مدیریت موجودی، مدیریت پرسنل و مشتریان دارد. این نرم‌افزار دارای دو سطح دسترسی کاربر و مدیر می‌باشد. در هر یک از شعب باید قابلیت ویرایش موجودی کالا، ویرایش یا افزودن مشتری جدید، ثبت و ویرایش فاکتور فروش وجود داشته باشد. تعریف پرسنل، کالاها، سطح دسترسی (کاربر و مدیر )توسط نرم افزار مدیریتی انجام می‌شود.
	\section{موارد استفاده}
	سرویس فروشگاه برای هر شعبه
	\begin{enumerate}
	\item اپراتور در نرم‌افزار فروشگاه با سطح دسترسی کاربر معمولی وارد می‌شود.
	\item احرازهویت ایشان توسط سرویس احرازهویت تصدیق می‌شود.
	\item اپراتور انتخاب‌های زیر را خواهد داشت:
	\begin{enumerate}
	\item افزودن و ویرایش مشتری جدید	
	\item افزودن، ویرایش، و حذف اقلام فاکتور فروش
	\item ثبت نهایی فروش و صدور فاکتور
	\end{enumerate}
	\end{enumerate}
	سرویس مدیریتی
	\begin{enumerate}
		\item کاربر با سطح دسترسی مدیر وارد می‌شود.
		\item احرازهویت وی توسط سرویس احرازهویت تصدیق می‌شود.
		\item مدیر قابلیت‌های زیر را خواهد داشت:
		\begin{enumerate}
			\item افزودن،ویرایش و حذف کاربر(اپراتور)
			\item افزودن، ویرایش و حذف کالا
			\item گزارشگیری‌های متنوع
			\item تمام توانایی کاربر معمولی
		\end{enumerate}
	\end{enumerate}
	%\textbf{حذف و ویرایش رکورد برای هر داده براساس شرایطی خاص انجام می‌شود.} \\
	دیگر سرویس‌ها که شامل سرویس مدیریت کاربران و مشتریان، سرویس مدیریت فروش و موجودی و کالا و مدیریت گزارش‌ها می‌شود توسط این دو سرویس مورد استفاده قرار خواهند گرفت.
	\section{پیشرانه‌های معماری (\lr{Architecture Drivers})}
	\begin{enumerate}
		\item \textbf{\lr{availability}}:
		سرویس‌دهی با بیشترین دسترس‌پذیری (به ویژه در ساعات اوج تراکنش‌ها)
		\item \textbf{\lr{performance}}:
		پاسخ درخواست‌ها در بازه محدود و مشخص داده شود
		\item \textbf{\lr{testability}}:
		قابلیت نمایش خطای سیستم برای تسهیل در تصحیح خطای
		\item \textbf{\lr{scalability}}:
		قابلیت افزودن شعب بدون نیاز به تغییرات عمده را داشته باشیم
		
	\end{enumerate}
	\newpage
	\section{سناریوها (\lr{Scenarios})}
	سناریوهای دسترس پذیری نرم افزار عبارتند از: 
	\begin{enumerate}
		\item سامانه باید در ساعات کاری به برای انجام عملیات‌هایی همچون ثبت تراکنش‌ها در دسترس باشد.
		\item در ساعات اوج مصرف سامانه باید دردسترس باشد
		\item در هر زمان برای گزارش‌گیری باید سامانه در دسترس باشد
	\end{enumerate}
\begin{table}[h]
	\centering
	\caption{سناریو \lr{availability}}
	\begin{tabular}{|c|c|}
		\hline
		\textbf{بخش‌های سناریو} & \textbf{توضیحات} \\
		\hline
		\lr{stimulus sources} & \lr{user, software, hardware}\\
		\lr{stimulus} & \lr{fault in any services} \\
		\lr{environment} &  \lr{peak hours} \\
		\lr{artifact} & \lr{store services ,order API, order services,report services}\\
		\lr{response} & \lr{1-prevent 2-detect 3-recovery}  \\
		\lr{response measures} & \lr{Time to detect fault, Avg availability in specific duration} \\
		\hline
	\end{tabular}
\end{table}
سناریوهای کارایی نرم افزار عبارتند از: 
\begin{enumerate}
	\item در زمان اوج فروش در شعب باید حداکثر تاخیر قابل قبولی داشته باشیم.
\end{enumerate}
\begin{table}[h]
	\centering
	\caption{سناریو \lr{performance}}
	\begin{tabular}{|c|c|}
		\hline	
		\textbf{بخش‌های سناریو} & \textbf{توضیحات} \\
		\hline	
\lr{stimulus sources} & \lr{services and user}\\
\lr{stimulus} & \lr{order} \\
\lr{environment} &  \lr{ordering in peak time} \\
\lr{artifact} & \lr{ordering, inventory, product,}\\
\lr{response} & \lr{fast, handle transaction when disconnecting}  \\
\lr{response measures} & \lr{response time, threshold of latency} \\
		\hline
	\end{tabular}
\end{table}
سناریوهای تست پذیری نرم افزار عبارتند از: 
\begin{enumerate}
	\item درصورت بروز خطا در سیستم قابلیت کشف علت و تصحیح آن مشخص باشد.
\end{enumerate}
\begin{table}[h!]
	\centering
	\caption{سناریوهای \lr{testability}}
	\begin{tabular}{|c|c|}
		\hline	
		\textbf{بخش‌های سناریو} & \textbf{توضیحات} \\
		\hline	
		\lr{stimulus sources} & \lr{users, unit or integrate test}\\
		\lr{stimulus} & \lr{a set of tests} \\
		\lr{environment} &  \lr{design \& development \& compile \& integration time} \\
		\lr{artifact} & \lr{the part of system}\\
		\lr{response} & \lr{execute test suit and capture result, activity which created fault}  \\
		\lr{response measures} & \lr{effort to find a fault or class of faults} \\
		\hline
	\end{tabular}
\end{table}
سناریوهای مقیاس پذیری نرم افزار عبارتند از: 
\begin{enumerate}
	\item قابلیت افزودن نودهای جدید برای شعب جدید
	\item قابلیت افزودن نودهای جدید به منظور جلوگیری از تاخیر بیشتر
\end{enumerate}
\begin{table}[H]
	\centering
	\caption{سناریوهای \lr{scalability}}
	\begin{tabular}{|c|c|}
		\hline	
		\textbf{بخش‌های سناریو} & \textbf{توضیحات} \\
		\hline	
		\lr{stimulus sources} & \lr{manager or admin}\\
		\lr{stimulus} & \lr{new branch, increase in customers} \\
		\lr{environment} &  \lr{runtime} \\
		\lr{artifact} & \lr{servers \& services}\\
		\lr{response} & \lr{delay \& speed compensation by adding new node}  \\
		\lr{response measures} & \lr{The system must be able to handle many times the normal load.} \\
		\hline
	\end{tabular}	
\end{table}

\end{document}
