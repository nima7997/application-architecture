\documentclass[12pt,a4paper]{article}
\usepackage{geometry}
\geometry{a4paper, margin=2.5cm}
\usepackage{graphicx}
\usepackage{fancyhdr}
\usepackage{xepersian}
\settextfont{B Nazanin}
\usepackage{titlesec}
\usepackage{enumitem}
\usepackage{lipsum}


\pagestyle{fancy}
\fancyhf{}
\fancyhead[L]{\textbf{نام اعضای تیم:} سید حسین زراعتکار و نیما گمرکیان}
\fancyhead[R]{\textbf{موضوع پروژه:} معماری نرم‌افزار فروشگاه‌های زنجیره‌ای}
\fancyfoot[C]{\thepage}


\titleformat{\section}{\large\bfseries}{\thesection}{1em}{}
\titleformat{\subsection}{\normalsize\bfseries}{\thesubsection}{1em}{}


\usepackage{xcolor}
\usepackage[
colorlinks=true,
linkcolor=black,    
urlcolor=blue,       
citecolor=black
]{hyperref}


\makeatletter
\renewcommand{\@dotsep}{4.5}
\makeatother

\begin{document}
	\begin{figure}
		\centering
		\includegraphics[width=0.7\linewidth]{Image/Icon/KNTU}
		\label{fig:kntu}
	\end{figure}
	\begin{titlepage}
		\centering
		\vspace*{3cm}
		{\Huge \textbf{پلتفرم  مدیریت  فروشگاه‌های زنجیره‌ای}\par}
		\vspace{2cm}
		{\Large \textbf{درس:} معماری نرم‌افزار \par}
		\vspace{1cm}
		{\Large \textbf{استاد:} دکتر اثنی‌عشری \par}
		\vspace{1cm}
		{\Large \textbf{اعضای تیم:} سید حسین زراعتکار و نیما گمرکیان\par}
		\vfill
		{\large تاریخ: \today \par}
	\end{titlepage}
	

	\newpage
	

	\section{مقدمه}
	موضوع انتخابی، طراحی معماری پلتفرم مدیریت فروشگاه‌های زنجیره‌ای می‌باشد. این سامانه به منظور یکپارچه سازی دادگان فروشگاه زنجیره‌ای مورد استفاده قرار می‌گیرد. برای پیاده سازی معماری این سامانه از معماری سرویس گرا یا میکرو سرویس استفاده خواهیم کرد. بخش‌هایی که برای این پلتفرم کاندیدا هستند بخش مدیریت فروشگاه، مدیریت موجودی و محصولات، مدیریت کارمندان، مدیریت مشتریان، مدیریت فروش و پرداخت‌ها و مدیریت گزارشات می‌باشد. در ادامه به بررسی پیشرانه‌های معماری و سناریو‌ها و خواهیم پرداخت.
	
	\section{پیشرانه‌های معماری (\lr{Architecture Drivers})}
	
	\begin{enumerate}
		\item \textbf{\lr{availability}}:
		سامانه در ساعات کاری باید در دسترس باشد. همچنین در صورت افزوده شدن سامانه آنلاین باید درسترس پذیری نیز برای سامانه فراهم شود.
		\item \textbf{\lr{performance}}:
		امکان ثبت سفارشات به صورت گسترده فراهم باشد. سرعت پاسخگویی به ثبت سفارشات باید مطلوب باشد.
		\item \textbf{\lr{integration}}:
		در صورت ایحاد شعب دیگر و اضافه شدن سرویس‌های مدیریت فروشگاه دیگر، قابلیت افزودن آن به سامانه بدون نیاز به تغییرات بنیادی فراهم باشد. همچنین در صورت نیاز بتوان بخش حسابداری، تامین و بخش‌های دیگر را به این پروژه متصل کرد.
	\end{enumerate}
	
	\section{سناریوها (\lr{Scenarios})}
\begin{table}[h]
	\centering
	\caption{سناریوهای \lr{availability} به صورت کلی}
	\begin{tabular}{|c|c|}
		\hline
		بخش‌های سناریو& توضیحات \\
		\hline
		منبع محرک & اپراتور فروشگاه، سخت افزار، نرم افزار و زیرساخت فیزیکی \\
		محرک & \lr{fault: crash, incorrect response} \\
		محیط & هنگام اجرای \lr{Crud}، ارسال و دریافت درخواست از سرویس‌ها، هنگام تعمییر سرویس ، در زمان پیک ثبت تراکنش  \\
		پاسخ & 1- جلوگیری از \lr{fault} 2- تشخیص \lr{fault} 3- بازیابی سرویس پس از \lr{fault} \\
		معیار سنجش & میزان زمان برای تشخیص و بازیابی \lr{fault}، میزان زمان در دسترس بودن سیستم به صورت میانگین \\
		\hline
	\end{tabular}
\end{table}
سناریوهای مختص در دسترس پذیری این نرم افزار عبارتند از: 
\begin{enumerate}
	\item افزایش ترافیک در زمان‌های خاص (روزهای خاص، تخفیفات و ...)
	\item  خرابی یکی از سرویس‌های حیاتی
\end{enumerate}

\begin{table}[h]
	\centering
	\caption{سناریوهای \lr{performance}}
	\begin{tabular}{|c|c|}
		\hline	
بخش‌های سناریو& توضیحات \\
		\hline	
منبع محرک & سرویس‌های داخلی و خارجی\\	
محرک & رخدادهای تصادفی درخواست‌ها و پاسخشان از سرویس‌های مختلف	\\	
محیط & درحالت: عادی، اضطراری، سربار زیاد \\	
پاسخ & استفاده از سرورهای پشتیبان، استفاده از \lr{timeout} 	\\	
معیار سنجش & تاخیر، توان عملیاتی، \lr{jitter} و...	\\	
		\hline
	\end{tabular}
\end{table}
\newpage
سناریوهای مختص در کارایی این نرم افزار عبارتند از: 
\begin{enumerate}
	\item  مدیریت افزایش بار در ساعات پیک خرید از فروشگاه
	(ثبت تراکنش‌های همزمان از چند صد فروشگاه)
	\item انجام گزارش‌گیری با زمان تقریبی مناسب
\end{enumerate}
\begin{table}[h]
	\centering
	\caption{سناریوهای \lr{integration}}
	\begin{tabular}{|c|c|}
		\hline	
		بخش‌های سناریو& توضیحات \\
		\hline	
				منبع محرک &  مدیرسامانه، ادمین، توسعه دهنده \\	
				محرک & توسعه نرم افزار برای مدیریت بهتر روی بخش‌ها به صورت یکپارچه،	\\	
				محیط & در رمان اجرا \\	
				پاسخ & تخصیص و آزادسازی سرور، تغییر سطح سرویس	\\	
				معیار سنجش & میزان تطابق با سرویس‌های جدید	\\	
		\hline
	\end{tabular}
\end{table}

سناریوهای مختص در یکپارچه‌سازی این نرم افزار عبارتند از: 
\begin{enumerate}
	\item قابلیت اتصال به سرویس‌هایی مانند حسابداری، مدیریت تامیین بدون نیاز به تغییر ماهیت سرویس‌ها
	\item در صورت تغییر سرویس‌های فروش و مدیریت کالا سرویس گزارشات نیاز به تغییر نداشته باشد و سرویس‌ها بتوانند با استفاده از روش‌هایی مانند استاندارد سازی یکپارچه شوند(این مورد می‌تواند با معیار \lr{modifiability} مشترک باشد)
\end{enumerate}
\textbf{توضیحات تکمیلی}\\
در این سامانه، ما چندین سرویس مدیریت فروشگاه داشته که برای هر یک از شعب فروشگاه به صورت مستقل کار می‌کند. همچنین برای سرویس محصولات و موجودی ما سرویسی داشته که موظف به مدیریت موجودی کالا و مدیریت بخش کالا را برعهده خواهد داشت. سرویس کارمندان و مشتریان نیز برای مدیریت اطلاعات مشتریان همچون اطلاعات پرسنلی، سبد‌های خرید و مواردی از این قبیل می‌باشد و از طرفی برای کارمندان به منظور مدیریت تراکنش‌ها نیز سرویس مستقل دیگری خواهیم داشت. سرویس احراز هویت نیز به منظور اعمال موارد امنیتی استفاده می‌شود. سرویس مدیریت گزارش‌ها به منظور گزارش گیری برای ادمین‌های سامانه و مدیران سامانه ایجاد شده است که دسترسی کاربران با سطح دسترسی پایین (اپراتورها) اجازه دسترسی به به این سرویس را ندارند.
\begin{figure}[h]
	\centering
	\includegraphics[width=0.7\linewidth]{image/basic-diagram}
	\caption{دیاگرام ساده‌ای برای نمایش اولیه سرویس‌ها}
	\label{fig:basic-diagram}
\end{figure}


\end{document}
