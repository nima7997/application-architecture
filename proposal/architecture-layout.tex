\documentclass[12pt,a4paper]{article}
\usepackage{geometry}
\geometry{a4paper, margin=2.5cm}
\usepackage{graphicx}
\usepackage{fancyhdr}
\usepackage{xepersian}
\settextfont{B Nazanin}
\usepackage{titlesec}
\usepackage{enumitem}
\usepackage{lipsum}
\usepackage{float}


\pagestyle{fancy}
\fancyhf{}
\fancyhead[L]{\textbf{نام اعضای تیم:} سید حسین زراعتکار و نیما گمرکیان}
\fancyhead[R]{\textbf{موضوع پروژه:} معماری نرم‌افزار فروشگاه‌های زنجیره‌ای}
\fancyfoot[C]{\thepage}


\titleformat{\section}{\large\bfseries}{\thesection}{1em}{}
\titleformat{\subsection}{\normalsize\bfseries}{\thesubsection}{1em}{}


\usepackage{xcolor}
\usepackage[
colorlinks=true,
linkcolor=black,    
urlcolor=blue,       
citecolor=black
]{hyperref}


\makeatletter
\renewcommand{\@dotsep}{4.5}
\makeatother

\begin{document}
	\begin{figure}
		\centering
		\includegraphics[width=0.7\linewidth]{Image/Icon/KNTU}
		\label{fig:kntu}
	\end{figure}
	\begin{titlepage}
		\centering
		\vspace*{3cm}
		{\Huge \textbf{پلتفرم  مدیریت  فروشگاه‌های زنجیره‌ای}\par}
		\vspace{2cm}
		{\Large \textbf{درس:} معماری نرم‌افزار \par}
		\vspace{1cm}
		{\Large \textbf{استاد:} دکتر اثنی‌عشری \par}
		\vspace{1cm}
		{\Large \textbf{اعضای تیم:} سید حسین زراعتکار و نیما گمرکیان\par}
		\vfill
		{\large تاریخ: \today \par}
	\end{titlepage}
	

	\newpage
	

	\section{مقدمه}
	این سامانه به منظور استفاده در فروشگاه‌های زنجیره‌ای برای مدیریت محصولات، موجودی، مدیریت فروش طراحی خواهد شد. این سامانه نیاز به یکپارچه‌سازی به منظور مدیریت شعب، کالا، فروش، موجودی، کارمندان(ثبت کننده تراکنش‌ها) دارد. این نرم‌افزار دارای دو سطح دسترسی کاربر و مدیر می‌باشد. در هر یک از شعب باید قابلیت ورود اپراتور، ویرایش موجودی کالا(افزایش موجودی)، ثبت و ویرایش فاکتور فروش وجود داشته باشد. تعریف پرسنل، کالاها، سطح دسترسی (کاربر و مدیر )توسط نقش مدیریتی انجام می‌شود. این سامانه در ساعاتی از روز بار کاری نسبتا سنگینی خواهد داشت، احراز هویت کاربران، ثبت سفارشات، تغییرات در سفارشات، افزایش موجودی کالا‌ها برای هر فروشگاه و موارد دیگر.
	\section{موارد استفاده}
	کاربر معمولی
	\begin{enumerate}
	\item اپراتور در نرم‌افزار فروشگاه با سطح دسترسی کاربر معمولی وارد می‌شود.
	\item احرازهویت ایشان توسط سرویس احرازهویت تصدیق می‌شود.
	\item اپراتور انتخاب‌های زیر را خواهد داشت:
	\begin{enumerate}
%	\item افزودن و ویرایش مشتری جدید	
	\item افزودن، ویرایش، و حذف اقلام فاکتور فروش
	\item ویرایش موجودی کالاها
	\end{enumerate}
	\end{enumerate}
	کاربر مدیر
	\begin{enumerate}
		\item کاربر با سطح دسترسی مدیر وارد می‌شود.
		\item احرازهویت وی توسط سرویس احرازهویت تصدیق می‌شود.
		\item مدیر قابلیت‌های زیر را خواهد داشت:
		\begin{enumerate}
			\item افزودن،ویرایش و حذف اپراتور
			\item افزودن، ویرایش و حذف کالا،موجودی،قیمت  
			\item گزارشگیری‌های متنوع
			\item  توانایی کاربر معمولی
		\end{enumerate}
	\end{enumerate}
	%\textbf{حذف و ویرایش رکورد برای هر داده براساس شرایطی خاص انجام می‌شود.} \\
	دیگر سرویس‌ها مانند مدیریت فروش و موجودی و کالا و مدیریت گزارش‌ها می‌شود توسط این دو سرویس مورد استفاده قرار خواهند گرفت.
	
	\section{پیشرانه‌های معماری (\lr{Architecture Drivers})}
	پیشرانه‌های مدنظر:
	\begin{enumerate}
		\item \textbf{\lr{availability}}
		%سرویس‌دهی با بیشترین دسترس‌پذیری (به ویژه در ساعات اوج تراکنش‌ها)
		\item \textbf{\lr{performance}}
	%	پاسخ درخواست‌ها در بازه محدود و مشخص داده شود
		\item \textbf{\lr{testability}}
	%	قابلیت نمایش خطای سیستم برای تسهیل در تصحیح خطای
		
	\end{enumerate}
	برخی از پیشرانه‌های دیگر که می‌توان در نظر گرفت:
	\begin{enumerate}
		\item \textbf{\lr{security}}
		\item \textbf{\lr{energy efficiency}}		
		\item \textbf{\lr{usability}}
		
		
	\end{enumerate}

	\section{سناریوها (\lr{Scenarios})}
	\textbf{\lr{availability scenario}}
	\begin{enumerate}
		\item سامانه باید دسترس پذیری بالایی داشته باشد زیرا ثبت تراکنش برای فروشگاه نیاید بیش از 5 ثانیه باشد پس در صورت خرابی سرویس دیگری باید سریعا جایگزین شود.
		\item  در صورت مشکل برای ماژول‌های ثبت تراکنش باید سریع جایگزینی برای آن انتخاب شود تا دسترس پذیری از بین نرود \\
		\begin{table}[h]
			\centering
			\caption{سناریوی دسترس‌پذیری}
			\begin{tabular}{|c|p{10cm}|}
				\hline
				\textbf{بخش‌های سناریو} & \textbf{توضیحات} \\
				\hline
				\lr{stimulus source} & فروشنده شعبه \\
				\hline
				\lr{stimulus} & از کار افتادن سرویس ثبت تراکنش و یا سرویس احراز هویت \\
				\hline
				\lr{environment} & ساعات اوج فروش \\
				\hline
				\lr{artifact} & سرویس ثبت تراکنش در سامانه مرکزی ، سرویس احراز هویت \\
				\hline
				\lr{response} & شناسایی خطا، ثبت لاگ، هدایت درخواست‌ها به سرویس سالم \\
				\hline
				\lr{response measure} & 
				زمان تشخیص خطا کمتر 15 ثانیه، 
				زمان بازیابی کمتر از 1 دقیقه پس از تشخیص، 
				دسترس پذیری \% 9.99 \\
				\hline
			\end{tabular}
		\end{table}
	\end{enumerate}
\textbf{\lr{performance scenario}}
\begin{enumerate}
	\item در زمان اوج ثبت سفارشات تاخیر برای ثبت تراکنش باید کمتر از 5 ثانیه باشد.
	
\end{enumerate}
\begin{table}[h]
	\centering
	\caption{سناریوی کارایی - ثبت تراکنش}
	\begin{tabular}{|c|p{10cm}|}
		\hline
		\textbf{بخش‌های سناریو} & \textbf{توضیحات} \\
		\hline
		\lr{stimulus source} & فروشنده شعبه \\
		\hline
		\lr{stimulus} & ارسال درخواست ثبت تراکنش \\
		\hline
		\lr{environment} & ساعات اوج فروش \\
		\hline
		\lr{artifact} & سرویس انبار و حسابداری \\
		\hline
		\lr{response} & پردازش تراکنش و ارسال پاسخ \\
		\hline
		\lr{response measure} & 
		زمان پاسخ کمتر از 5 ثانیه \\
		\hline
	\end{tabular}
\end{table}

\textbf{\lr{testability scenario}}
\begin{enumerate}
	\item سامانه باید در زمان توسعه و تست قابلیت تست داشته باشد. و این قابلیت سبب تسریع در خطایابی و تصحیح خطا داشته باشد.
\end{enumerate}

\begin{table}[h]
	\centering
	\caption{سناریوی تست‌پذیری}
	\begin{tabular}{|c|p{10cm}|}
		\hline
		\textbf{بخش‌های سناریو} & \textbf{توضیحات} \\
		\hline
		\lr{stimulus source} & توسعه‌دهنده یا تست کننده \\
		\hline
		\lr{stimulus} & اجرای تست برای بررسی عملکرد یک سرویس \\
		\hline
		\lr{environment} & زمان توسعه ، تست  \\
		\hline
		\lr{artifact} & سرویس مد نظر \\
		\hline
		\lr{response} & اجرای تست و ارائه گزارش وضعیت داخلی \\
		\hline
		\lr{response measure} & 
		زمان شناسایی خطا کمتر از 10 دقیقه \\
		\hline
	\end{tabular}
\end{table}



\end{document}
